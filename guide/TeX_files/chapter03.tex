\chapter{Get Started with ROS}
In this chapter, we are going to explore what are nodes and topic inside of ROS.

\newpage
\section{Introduction to ROS}
 The Robot Operating System or ROS use a distributed publisher and subscriber system to pass message between nodes. The nodes can also be on another computer or local. By using this architecture, we are able to scale on multiple computers over our network. This also helps use to debug the robot by hooking our laptop to the robot network. 
 
 Another important part of ROS is the tools. The tools like RQT, RVIZ and roswtf, help us debug whats going on with the robot in real-time. This is the avantage by using a pre-build ecosystem to build our robot with.

\newpage

\section{Introduction to Nodes}
The nodes are the building blocks of any ROS based system. For our robot S.A.R.A. we use around 200 nodes when we are running fully autonomously. 

\newpage

\section{Introduction to Packages}
Each nodes must live inside of a directory that is call a packages. Each driver, state machine, and library. 

\newpage

\section{Navigating the ROS Filesystem}
\subsection{Prerequisites}
For this tutorial we will inspect a package in ros-tutorials, please install it using

\begin{lstlisting}[breaklines=True language=bash]
$ sudo apt-get install ros-kinetic-ros-tutorials
\end{lstlisting}

\subsection{Quick Overview of Filesystem Concepts}

\begin{itemize}
	\item Packages: Packages are the software organization unit of ROS code. Each package can contain libraries, executables, scripts, or other artifacts.
	\item Manifest(Package.xml): A manifest is a description of a package. It serves to define dependencies between packages and to capture meta information about the package like version, maintainer, license, etc...
\end{itemize}

\subsubsection{rospack}

rospack allows you to get information about packages. In this tutorial, we are only going to cover the find option, which returns the path to package. 

\begin{lstlisting}[breaklines=True language=bash]
$ rospack [packages_name]
\end{lstlisting}

Example:
\begin{lstlisting}[breaklines=True language=bash]
$ rospack roscpp
\end{lstlisting}

Would return
\begin{lstlisting}[breaklines=True language=bash]
	/opt/ros/kinetic/share/roscpp
\end{lstlisting}

\subsubsection{roscd}
roscd is part of the rosbash suite. It allows you to change directory (cd) directly to a package or a stack.

\begin{lstlisting}[breaklines=True language=bash]
$ roscd [packages_name]
\end{lstlisting}

Example:
\begin{lstlisting}[breaklines=True language=bash]
roscd roscpp
\end{lstlisting}

After we check what the current directory is:
\begin{lstlisting}[breaklines=True language=bash]
pwd
\end{lstlisting}

This command will return
\begin{lstlisting}[breaklines=True language=bash]
	/opt/ros/kinetic/share/roscpp
\end{lstlisting}


\subsubsection{rosls}
rosls is part of the rosbash suite. It allows you to ls directly in a package by name rather than by absolute path.

\begin{lstlisting}[breaklines=True language=bash]
$ rosls [locationname[/subdir]]
\end{lstlisting}

example:
\begin{lstlisting}[breaklines=True language=bash]
$ rosls roscpp_tutorials
\end{lstlisting}

would return:
\begin{lstlisting}[breaklines=True language=bash]
	cmake launch package.xml  srv
\end{lstlisting}

\subsubsection{review}
You may have noticed a pattern with the naming of the ROS tools:
\begin{itemize}
	\item rospack = ros + packages
	\item roscd = ros + cd
	\item rosls = ros + ls
\end{itemize}
This naming pattern holds for many of the ROS tools.Now that you can get around in ROS, let's create a package.

\newpage
\pagebreak

\section{Create a ros packages}
In this section, we are going to learn how to create a package. For this tutorial, we are going to be using the catkin\textunderscore ws we created before.

\subsection{Basic of anatomy of a ROS package}
For a package to be considered a catkin package it must meet a few requirements:
\begin{itemize}
	\item The package must contain a catkin compliant package.xml file.
	\item The package must contain a CMakeLists.txt which uses catkin.
	\item Each package must have its own folder
\end{itemize}

\noindent The simplest possible package might have a structure which looks like this:

\begin{lstlisting}[breaklines=true languages=bash]
my_package/
	CMakeLists.txt
	package.xml
\end{lstlisting}

\subsection{Creating a catkin package}

To create the package you must be inside you catkin\textunderscore ws
\begin{lstlisting}[breaklines=true languages=bash]
$ cd ~/catkin_ws/src
\end{lstlisting}

\noindent To create the package there is already a script inside of ros.

\begin{lstlisting}[breaklines=true languages=bash]
$ catkin_create_pkg beginner_tutorials std_msgs rospy roscpp
\end{lstlisting}

\noindent This will create a beginner\textunderscore tutorials folder which contains a package.xml and a CMakeLists.txt, which have been partially filled out with the information you gave catkin\textunderscore create\textunderscore pkg.
\newline

\noindent catkin\textunderscore create\textunderscore pkg requires that you give it a package\textunderscore name and optionally a list of dependencies on which that package depends:

\begin{lstlisting}[breaklines=true languages=bash]
# This is an example, do not try to run this
# catkin_create_pkg <package_name> [depend1] [depend2] [depend3]
\end{lstlisting}
\newpage
\pagebreak
\subsection{Building a catkin workspace}
Now you need to build the packages in the catkin workspace:

\begin{lstlisting}[breaklines=true languages=bash]
$ cd ~/catkin_ws
$ catkin_make
\end{lstlisting}

\noindent This will build the whole workspace.

\subsection{Customizing Your Package}
This part of the tutorial will look at each file generated by catkin\textunderscore create\textunderscore pkg and describe, line by line, each component of those files and how you can customize them for your package.

\subsubsection{Customizing the package.xml}
The generated package.xml should be in your new package. Now lets go through the new package.xml and touch up any elements that need your attention.
\newline
\newline
First update the description tag:

\begin{lstlisting}[breaklines=false languages=xml]
<description>The beginner_tutorials package</description>
\end{lstlisting}
Change the description to anything you like, but by convention the first sentence should be short while covering the scope of the package. If it is hard to describe the package in a single sentence then it might need to be broken up.
\newline
\pagebreak
\newpage
Next comes the maintainer tag:
\begin{lstlisting}[breaklines=true languages=xml]
 <!-- One maintainer tag required, multiple allowed, one person per tag --> 
 <!-- Example:  -->
 <!-- <maintainer email="jane.doe@example.com">Jane Doe</maintainer> -->
 <maintainer email="user@todo.todo">user</maintainer>
\end{lstlisting}

\noindent This is a required and important tag for the package.xml because it lets others know who to contact about the package. At least one maintainer is required, but you can have many if you like. The name of the maintainer goes into the body of the tag, but there is also an email attribute that should be filled out:

\begin{lstlisting}[breaklines=true languages=xml]
<maintainer email="you@yourdomain.tld">Your Name</maintainer>
\end{lstlisting}

\noindent Next is the license tag, which is also required:
\begin{lstlisting}[breaklines=true languages=xml]
 <!-- One license tag required, multiple allowed, one license per tag -->
 <!-- Commonly used license strings: -->
 <!--   BSD, MIT, Boost Software License, GPLv2, GPLv3, LGPLv2.1, LGPLv3 -->
 <license>TODO</license>
\end{lstlisting}

\noindent You should choose a license and fill it in here. Some common open source licenses are BSD, MIT, Boost Software License, GPLv2, GPLv3, LGPLv2.1, and LGPLv3.
\newline

\noindent The one that we are going to use the most is Apache2.0 (\url{https://www.apache.org/licenses/LICENSE-2.0})
\newline

\noindent So for this package, we are going to use Apache2.0.

\begin{lstlisting}[breaklines=true languages=xml]
 <license>Apache2.0</license>
\end{lstlisting}

\newpage

\noindent The next set of tags describe the dependencies of your package. The dependencies are split into build\_depend, buildtool\_depend, run\_depend, test\_depend.

\begin{lstlisting}[breaklines=true languages=xml]
<!-- The *_depend tags are used to specify dependencies -->
<!-- Dependencies can be catkin packages or system dependencies -->
<!-- Examples: -->
<!-- Use build_depend for packages you need at compile time: -->
<!--   <build_depend>genmsg</build_depend> -->
<!-- Use buildtool_depend for build tool packages: -->
<!--   <buildtool_depend>catkin</buildtool_depend> -->
<!-- Use run_depend for packages you need at runtime: -->
<!--   <run_depend>python-yaml</run_depend> -->
<!-- Use test_depend for packages you need only for testing: -->
<!--   <test_depend>gtest</test_depend> -->
<buildtool_depend>catkin</buildtool_depend>
<build_depend>roscpp</build_depend>
<build_depend>rospy</build_depend>
<build_depend>std_msgs</build_depend>
\end{lstlisting}

All of our listed dependencies have been added as a build\_depend for us, in addition to the default buildtool\_depend on catkin. In this case we want all of our specified dependencies to be available at build and run time, so we'll add a run\_depend tag for each of them as well:

\begin{lstlisting}[breaklines=true languages=xml]
 <buildtool_depend>catkin</buildtool_depend>
 
 <build_depend>roscpp</build_depend>
 <build_depend>rospy</build_depend>
 <build_depend>std_msgs</build_depend>
 
 <run_depend>roscpp</run_depend>
 <run_depend>rospy</run_depend>
 <run_depend>std_msgs</run_depend>
\end{lstlisting}
\pagebreak
As you can see the final package.xml, without comments and unused tags, is much more concise:


\begin{lstlisting}[breaklines=true languages=xml]
<?xml version="1.0"?>
<package>
<name>beginner_tutorials</name>
<version>0.1.0</version>
<description>The beginner_tutorials package</description>

<maintainer email="you@yourdomain.tld">Your Name</maintainer>
<license>BSD</license>
<url type="website">http://wiki.ros.org/beginner_tutorials</url>
<author email="you@yourdomain.tld">Jane Doe</author>

<buildtool_depend>catkin</buildtool_depend>

<build_depend>roscpp</build_depend>
<build_depend>rospy</build_depend>
<build_depend>std_msgs</build_depend>

<run_depend>roscpp</run_depend>
<run_depend>rospy</run_depend>
<run_depend>std_msgs</run_depend>

</package>
\end{lstlisting}
\pagebreak
\newpage

\subsection{Building your Package}
\subsubsection{Using catkin\_make}
catkin\_make is a command line tool which adds some convenience to the standard catkin workflow. You can imagine that catkin\_make combines the calls to cmake and make in the standard CMake workflow.
\newline
\newline
Usage:
\begin{lstlisting}[breaklines=true languages=bash]
# In a catkin workspace
$ catkin_make [make_targets] [-DCMAKE_VARIABLES=...]
\end{lstlisting}

\noindent catkin projects can be built together in workspaces. Building zero to many catkin packages in a workspace follows this work flow:

\begin{lstlisting}[breaklines=true languages=bash]
# In a catkin workspace
$ catkin_make
\end{lstlisting}

By doing this inside of your workspace, you now have a build packages.

\section{Understanding ROS node}

\subsection{Prerequisites}
For this tutorial, we will use a lightweight simulator, please install it using
\begin{lstlisting}[breaklines=true languages=bash]
$ sudo apt-get install ros-kinetic-ros-tutorials
\end{lstlisting}

\subsection{Quick Overview of Graph Concepts}

\begin{itemize}
	\item Nodes: A node is an executable that uses ROS to communicate with other nodes.
	\item Messages: ROS data type used when subscribing or publishing to a topic
	\item Topic: Nodes can publish messages to a topic as well as subscibe to a topic to receive messages.
	\item Master: Name service for ROS (i.e. helps nodes find each other)
	\item rosout: ROS equivalent of stdout/stderr
	\item roscore: Master + rosout +parameter server
\end{itemize}

\subsection{Nodes}
A node really isn't much more than an executable file within a ROS package. ROS nodes use a ROS client library to communicate with other nodes. Nodes can publish or subscribe to a Topic. Nodes can also provide or use a Service.

\subsection{Client Libraries}
Ros client libraries allow nodes written in different programming languages to communicate:
\begin{itemize}
	\item rospy = python client library
	\item roscpp = c++ client library
\end{itemize}

\subsection{roscore}

roscore is the first thing you should run when using ROS.

Please run:
\begin{lstlisting}[breaklines=true languages=bash]
$ roscore
\end{lstlisting}

You will see somethiong similar to:

\begin{lstlisting}[breaklines=true languages=bash]
... logging to ~/.ros/log/9cf88ce4-b14d-11df-8a75-00251148e8cf/roslaunch-machine_name-13039.log
Checking log directory for disk usage. This may take awhile.
Press Ctrl-C to interrupt
Done checking log file disk usage. Usage is <1GB.

started roslaunch server http://machine_name:33919/
ros_comm version 1.4.7

SUMMARY
======

PARAMETERS
* /rosversion
* /rosdistro

NODES

auto-starting new master
process[master]: started with pid [13054]
ROS_MASTER_URI=http://machine_name:11311/

setting /run_id to 9cf88ce4-b14d-11df-8a75-00251148e8cf
process[rosout-1]: started with pid [13067]
started core service [/rosout]
\end{lstlisting}

\subsection{Using rosnode}
Open up a new terminal, and let's use rosnode to see what running roscore did... Bare in mind to keep the previous terminal open either by opening a new tab or simply minimizing it.

rosnode displays information about the ROS nodes that are currently running. The rosnode list command lists these active nodes:

\begin{lstlisting}[breaklines=true languages=bash]
$ rosnode list
\end{lstlisting}

you will see:
\begin{lstlisting}[breaklines=true languages=bash]
/rosout
\end{lstlisting}

This showed us that here is only one node runnning: rosout. This is always running as it collects and logs nodes debugging output.

\subsection{Using rosrun}
rosrun allows you to use the package name to directly run a node within a package (without having to know the package path)

Usage:
\begin{lstlisting}[breaklines=true languages=bash]
$ rosrun [package_name] [node_name]
\end{lstlisting}

So now we can run the turtlesim\_node in the turtlesim package.

Then, in a new terminal:

\begin{lstlisting}[breaklines=true languages=bash]
$ rosrun turtlesim turtlesim_node
\end{lstlisting}

You will see the turtlesim window.

In a new terminal:

\begin{lstlisting}[breaklines=true languages=bash]
rosnode list
\end{lstlisting}

you will see something similar to:
\begin{lstlisting}[breaklines=true languages=bash]
/rosout
/turtlesim
\end{lstlisting}

\section{Understanding ROS Topics}
\subsection{turtle keyboard teleoperation}
We'll also need something to drive the turtle around with. Please run in a new terminal:

\begin{lstlisting}[breaklines=true languages=bash]
$ rosrun turtlesim turtlesim_node
\end{lstlisting}

Now you can use the arrow keys of the keyboard to drive the turtle around. If you can not drive the turtle select the terminal window of the turtle\_teleop\_key to make sure that the keys that you type are recorded.

\subsection{ROS Topics}
The turtlesim\_node and the turtle\_teleop\_key node are communicating with each other over a ROS Topic. The turtle\_teleop\_key is publishing the key strokes on a topic, while turtlesim subscribes to the same topic to receive the key strokes. Let's use rqt\_graph which shows the nodes and topics currently running.

\subsubsection{Using rqt\_graph}
rqt\_graph creates a dynamic graph of what's going on in the system. rqt\_graph is part or the rqt package. Unless you already have it installed, run:
\begin{lstlisting}[breaklines=true languages=bash]
$ sudo apt-get install ros-kinetic-rqt
$ sudo apt-get install ros-kinetic-rqt-common-plugins
\end{lstlisting}

In a new terminal:

\begin{lstlisting}[breaklines=true languages=bash]
$ rosrun rqt_graph rqt_graph
\end{lstlisting}

from this view you can see how the nodes communicate with each other.

\section{Introducing rostopic}
The rostopic tool allows you to get the information about ROS topics.

You can use the help option to get the available sub-commands for rostopic.
\begin{lstlisting}[breaklines=true languages=bash]
$ rostopic -h
\end{lstlisting}

\begin{lstlisting}[breaklines=true languages=bash]
rostopic bw     display bandwidth used by topic
rostopic echo   print messages to screen
rostopic hz     display publishing rate of topic    
rostopic list   print information about active topics
rostopic pub    publish data to topic
rostopic type   print topic type
\end{lstlisting}

\subsubsection{Using rostopic echo}
rostopic echo shows the data published on a topic.

Usage:
\begin{lstlisting}[breaklines=true languages=bash]
rostopic echo [topic]
\end{lstlisting}

Let's look at the command velocity data published by the turtle\_teleop\_key node.

\begin{lstlisting}[breaklines=true languages=bash]
$ rostopic echo /turtle1/cmd_vel
\end{lstlisting}

You probably won't see anything happen because no data is being published on the topic. Let's make turtle\_teleop\_key publish data by pressing the arrow keys. Remember if the turtle isn't moving you need to select the turtle\_teleop\_key terminal again.

You should now see the following when you press the up key:
\begin{lstlisting}[breaklines=true languages=bash]
linear: 
x: 2.0
y: 0.0
z: 0.0
angular: 
x: 0.0
y: 0.0
z: 0.0
---
linear: 
x: 2.0
y: 0.0
z: 0.0
angular: 
x: 0.0
y: 0.0
z: 0.0
---
\end{lstlisting}

\subsubsection{Using rostopic list}
rostopic list returns a list of all topics currently subscribed to and published.

Let's figure out what argument the list sub-commands needs. In a new terminal run:
\begin{lstlisting}[breaklines=true languages=bash]
$ rostopic list -h
\end{lstlisting}

\begin{lstlisting}[breaklines=true languages=bash]
Usage: rostopic list [/topic]

Options:
-h, --help            show this help message and exit
-b BAGFILE, --bag=BAGFILE
list topics in .bag file
-v, --verbose         list full details about each topic
-p                    list only publishers
-s                    list only subscribers
\end{lstlisting}

For rostopic list use the verbose option:
\begin{lstlisting}[breaklines=true languages=bash]
rostopic list -v
\end{lstlisting}

This displays a verbose list of topics to publish to and subscribe to and their type.

\begin{lstlisting}[breaklines=true languages=bash]
Published topics:
* /turtle1/color_sensor [turtlesim/Color] 1 publisher
* /turtle1/cmd_vel [geometry_msgs/Twist] 1 publisher
* /rosout [rosgraph_msgs/Log] 2 publishers
* /rosout_agg [rosgraph_msgs/Log] 1 publisher
* /turtle1/pose [turtlesim/Pose] 1 publisher

Subscribed topics:
* /turtle1/cmd_vel [geometry_msgs/Twist] 1 subscriber
* /rosout [rosgraph_msgs/Log] 1 subscriber
\end{lstlisting}

\section{ROS Messages}
Communication on topics happens by sending ROS messages between nodes. For the publisher (turtle\_teleop\_key) and subscriber(turtlesim\_node) to communicate, the publisher and subscriber must send and receive the same type of message. This means that a topic type is defined by the message type published on it. The type of the message sent on a topic can be determined using rostopic type.

\subsection{Using rostopic type}
rostopic type returns the message type of any topic being published.

Usage:
\begin{lstlisting}[breaklines=true language=bash]
rostopic type [topic]
\end{lstlisting}

Try:
\begin{lstlisting}[breaklines=true languages=bash]
$ rostopic type /turtle1/cmd_vel
\end{lstlisting}

You should get:
\begin{lstlisting}[breaklines=true language=bash]
geometry_msgs/Twist
\end{lstlisting}

We can look at the details of the message using rosmsg:

\begin{lstlisting}[breaklines=true language=bash]
$ rosmsg show geometry_msgs/Twist
\end{lstlisting}

You will get this definition:

\begin{lstlisting}[breaklines=true language=bash]
geometry_msgs/Vector3 linear
float64 x
float64 y
float64 z
geometry_msgs/Vector3 angular
float64 x
float64 y
float64 z
\end{lstlisting}

\section{rostopic continued}
Now that we have learned about ROS messages, let's use rostopic with messages.

\subsection{Using rostopic pub}
rostopic pub publishes data on to a topic currently advertised.

Usage:
\begin{lstlisting}[breaklines=true languages=bash]
rostopic pub [topic] [msg_type] [args]
\end{lstlisting}

example:
\begin{lstlisting}[breaklines=true langauges=bash]
$ rostopic pub -1 /turtle1/cmd_vel geometry_msgs/Twist -- '[2.0, 0.0, 0.0]' '[0.0, 0.0, 1.8]'
\end{lstlisting}

The previous command will send a single message to turtlesim telling it to move with an linear velocity of 2.0, and an angular velocity of 1.8 .

This is a pretty complicated example, so lets look at each argument in detail.

This command will publish messages to a given topic:
\begin{lstlisting}[breaklines=true language=bash]
rostopic pub
\end{lstlisting}

This option (dash-one) causes rostopic to only publish one message then exit:
\begin{lstlisting}[breaklines=true language=bash]
-1
\end{lstlisting}

This is the name of the topic to publish to:
\begin{lstlisting}[breaklines=true language=bash]
/turtle1/cmd_vel
\end{lstlisting}

This is the message type to use when publishing to the topic:
\begin{lstlisting}[breaklines=true languages=bash]
geometry_msgs/Twist
\end{lstlisting}

the option(double-dash) tells the option parser that none of the following arguments is an option. This is required in cases where your arguments have a leading dash -, like negative numbers.
\begin{lstlisting}[breaklines=true languages=bash]
--
\end{lstlisting}

As noted before, a geometry\_msgs/Twist msg has two vectors of three floating point elements each: linear and angular. In this case, '[2.0, 0.0, 0.0]' becomes the linear value with x=2.0, y=0.0, and z=0.0, and '[0.0, 0.0, 1.8]' is the angular value with x=0.0, y=0.0, and z=1.8.
\begin{lstlisting}[breaklines=true languages=yaml]
'[2.0, 0.0, 0.0]' '[0.0, 0.0, 1.8]' 
\end{lstlisting}

You may have noticed that the turtle has stopped moving; this is because the turtle requires a steady stream of commands at 1 Hz to keep moving. We can publish a steady stream of commands using rostopic pub -r command:

\begin{lstlisting}[breaklines=true languages=bash]
$ rostopic pub /turtle1/cmd_vel geometry_msgs/Twist -r 1 -- '[2.0, 0.0, 0.0]' '[0.0, 0.0, -1.8]'
\end{lstlisting}

As you can see the turtle is running in a continuous circle. In a new terminal, we can use rostopic echo to see the data published by our turtlesim:

\begin{lstlisting}[breaklines=true languages=bash]
rostopic echo /turtle1/pose
\end{lstlisting}

\subsection{Using rostopic hz}

rostopic hz reports the rate at which data is published

Usage:
\begin{lstlisting}[breaklines=true languages=bash]
rostopic hz [topic]
\end{lstlisting}

Let's see how fast the turtlesim\_node is publishing /turtle1/pose:
\begin{lstlisting}[breaklines=true languages=bash]
$ rostopic hz /turtle1/pose
\end{lstlisting}

You will see:
\begin{lstlisting}[breaklines=true languages=bash]
subscribed to [/turtle1/pose]
average rate: 59.354
min: 0.005s max: 0.027s std dev: 0.00284s window: 58
average rate: 59.459
min: 0.005s max: 0.027s std dev: 0.00271s window: 118
average rate: 59.539
min: 0.004s max: 0.030s std dev: 0.00339s window: 177
average rate: 59.492
min: 0.004s max: 0.030s std dev: 0.00380s window: 237
average rate: 59.463
min: 0.004s max: 0.030s std dev: 0.00380s window: 290
\end{lstlisting}

Now we can tell that the turtlesim is publishing data about our turtle at the rate of 60 Hz. We can also use rostopic type in conjunction with rosmsg show to get in depth information about a topic:

\begin{lstlisting}[breaklines=true languages=bash]
$ rostopic type /turtle1/cmd_vel | rosmsg show
\end{lstlisting}

\subsection{Using rqt\_plot}

rqt\_plot displays a scrolling time plot of the data published on topics. Here we'll use rqt\_plot to plot the data being published on the /turtle1/pose topic. First, start rqt\_plot by typing:

\begin{lstlisting}[breaklines=true languages=bash]
$ rosrun rqt_plot rqt_plot
\end{lstlisting}
In the new window that should pop up, a text box in the upper left corner gives you the ability to add any topic to the plot. Typing /turtle1/pose/x will highlight the plus button, previously disabled. Press it and repeat the same procedure with the topic /turtle1/pose/y. You will now see the turtle's x-y location plotted in the graph.

Pressing the minus button shows a menu that allows you to hide the specified topic from the plot. Hiding both the topics you just added and adding /turtle1/pose/theta will result in the plot shown in the next figure.

\section{Understanding ROS Services and Parameters}

\subsection{ROS Services}
Services are another way that nodes can communicate with each other. Services allow nodes to send a request and receive a response.

\subsection{Using rosservice}
rosservice can easily attach to ROS's client/service framework with services. rosservice has many commands that can be used on topics, as show below:

Usage:
\begin{lstlisting}[breaklines=true languages=bash]
rosservice list         print information about active services
rosservice call         call the service with the provided args
rosservice type         print service type
rosservice find         find services by service type
rosservice uri          print service ROSRPC uri
\end{lstlisting}

\subsubsection{rosservice list}

\begin{lstlisting}[breaklines=true languages=bash]
$ rosservice list
\end{lstlisting}

The list command shows us that the turtlesim node provides nine services: reset, clear, spawn, kill, turtle1/set\_pen, /turtle1/teleport\_absolute, /turtle1/teleport\_relative, turtlesim/get\_loggers, and turtlesim/set\_logger\_level. There are also two services related to the separate rosout node: /rosout/get\_loggers and /rosout/set\_logger\_level.

\begin{lstlisting}[breaklines=true languages=bash]
/clear
/kill
/reset
/rosout/get_loggers
/rosout/set_logger_level
/spawn
/teleop_turtle/get_loggers
/teleop_turtle/set_logger_level
/turtle1/set_pen
/turtle1/teleport_absolute
/turtle1/teleport_relative
/turtlesim/get_loggers
/turtlesim/set_logger_level
\end{lstlisting}

Let's look more closely at the clear service using rosservice type

\subsubsection{rosservice type}
Usage:

\begin{lstlisting}[breaklines=true languages=bash]
rosservice call [service] [args]
\end{lstlisting}

Here we'll call with no arguments because the service is of type empty:

\begin{lstlisting}[breaklines=true languages=bash]
$ rosservice call /clear
\end{lstlisting}

This does what we expect, it clears the background of the turtlesim\_node.

Let's look at the case where the service has arguments by looking at the information for the service spawn:

\begin{lstlisting}[breaklines=true languages=bash]
$ rosservice type /spawn | rossrv show
\end{lstlisting}

\begin{lstlisting}[breaklines=true languages=bash]
float32 x
float32 y
float32 theta
string name
---
string name
\end{lstlisting}

This service lets us spawn a new turtle at a given location and orientation. The name field is optional, so let's not give our new turtle a name and let turtlesim create one for us.

\begin{lstlisting}[breaklines=true languages=bash]
$ rosservice call /spawn 2 2 0.2 ""
\end{lstlisting}

The service call returns with the name of the newly created turtle
\begin{lstlisting}[breaklines=true languages=bash]
name: turtle2
\end{lstlisting}

Now we should have two turtles in our turtlesim.

\subsection{Using rosparam}
rosparam allows you to store and manipulate data on the ROS Parameter Server. The Parameter Server can store integers, floats, boolean, dictionaries, and lists. rosparam uses the YAML markup language for syntax. In simple cases, YAML looks very natural: 1 is an integer, 1.0 is a float, one is a string, true is a boolean, [1, 2, 3] is a list of integers, and {a: b, c: d} is a dictionary. rosparam has many commands that can be used on parameters, as shown below:

Usage:
\begin{lstlisting}[breaklines=true languages=bash]
rosparam set            set parameter
rosparam get            get parameter
rosparam load           load parameters from file
rosparam dump           dump parameters to file
rosparam delete         delete parameter
rosparam list           list parameter names
\end{lstlisting}

Let's look at what parameters are currently on the param server.

\subsubsection{rosparam list}

\begin{lstlisting}[breaklines=true languages=bash]
$ rosparam list
\end{lstlisting}

Here we can see that the turtlesim node has three parameters on the param server for the background color:

\begin{lstlisting}[breaklines=true languages=bash]
/background_b
/background_g
/background_r
/rosdistro
/roslaunch/uris/host_57aea0986fef__34309
/rosversion
/run_id
\end{lstlisting}

Let's change one of the parameter values using rosparam set:

\subsubsection{rosparam set and rosparam get}
Usage:
\begin{lstlisting}[breaklines=true languages=bash]
rosparam set [param_name]
rosparam get [param_name]
\end{lstlisting}

Here will change the red channel of the background color:

\begin{lstlisting}[breaklines=true languages=bash]
$ rosparam set /background_r 150
\end{lstlisting}
This changes the parameter value, now we have to call the clear service for the parameter change to take effect:

\begin{lstlisting}[breaklines=true languages=bash]
$ rosservice call /clear
\end{lstlisting}

Now let's look at the values of other parameters on the param server. Let's get the value of the green background channel:

\begin{lstlisting}[breaklines=true languages=bash]
$ rosparam get /background_g 
\end{lstlisting}

\begin{lstlisting}[breaklines=true languages=bash]
86
\end{lstlisting}

We can also use rosparam get / to show us the contents of the entire Parameter Server.

\begin{lstlisting}[breaklines=true languages=bash]
$ rosparam get /
\end{lstlisting}

\begin{lstlisting}[breaklines=true languages=bash]
background_b: 255
background_g: 86
background_r: 150
roslaunch:
uris: {'aqy:51932': 'http://aqy:51932/'}
run_id: e07ea71e-98df-11de-8875-001b21201aa8
\end{lstlisting}

You may wish to store this in a file so that you can reload it at another time. This is easy using rosparam:

\subsubsection{rosparam dump rosparam load}
Usage:

\begin{lstlisting}[breaklines=true languages=bash]
rosparam dump [file_name] [namespace]
rosparam load [file_name] [namespace]
\end{lstlisting}

Here we write all the parameters to the file params.yaml

\begin{lstlisting}[breaklines=true languages=bash]
$ rosparam dump params.yaml
\end{lstlisting}

You can even load these yaml files into a new namespaces, e.g. copy:
\begin{lstlisting}[breaklines=true languages=bash]
$ rosparam load params.yaml copy
$ rosparam get /copy/background_b
\end{lstlisting}

\begin{lstlisting}
255
\end{lstlisting}

Now that you understand how ROS services and params work, let's try using roslaunch.

\section{roslaunch}
roslaunch starts nodes as defined in a launch file.

Usage:
\begin{lstlisting}[breaklines=true languages=bash]
$ roslaunch [package] [filename.launch]
\end{lstlisting}

First go to the beginner\_tutorials package we created and built earlier:

\begin{lstlisting}[breaklines=true languages=bash]
$ roscd beginner_tutorials
\end{lstlisting}

Then let's make a launch directory:

\begin{lstlisting}[breaklines=true languages=bash]
$ mkdir launch
$ cd launch
\end{lstlisting}

\subsection{The launch file}

Now let's create a launch file called turtlemimic.launch and paste the following:

\begin{lstlisting}[breaklines=true languages=bash]
<launch>

<group ns="turtlesim1">
<node pkg="turtlesim" name="sim" type="turtlesim_node"/>
</group>

<group ns="turtlesim2">
<node pkg="turtlesim" name="sim" type="turtlesim_node"/>
</group>

<node pkg="turtlesim" name="mimic" type="mimic">
<remap from="input" to="turtlesim1/turtle1"/>
<remap from="output" to="turtlesim2/turtle1"/>
</node>

</launch>
\end{lstlisting}

\subsection{The Launch File Explained}

Now, let's break the launch xml down.

\begin{lstlisting}[breaklines=true languages=xml]
<launch>
\end{lstlisting}

Here we start the launch file with the launch tag, so that the file is identified as a launch file.

\begin{lstlisting}[breaklines=true languages=xml]
  <group ns="turtlesim1">
<node pkg="turtlesim" name="sim" type="turtlesim_node"/>
</group>

<group ns="turtlesim2">
<node pkg="turtlesim" name="sim" type="turtlesim_node"/>
</group>
\end{lstlisting}

Here we start two groups with a namespace tag of turtlesim1 and turtlesim2 with a turtlesim node with a name of sim. This allows us to start two simulators without having name conflicts.

\begin{lstlisting}[breaklines=true languages=xml]
<node pkg="turtlesim" name="mimic" type="mimic">
	<remap from="input" to="turtlesim1/turtle1"/>
	<remap from="output" to="turtlesim2/turtle1"/>
</node>
\end{lstlisting}

Here we start the mimic node with the topics input and output renamed to turtlesim1 and turtlesim2. This renaming will cause turtlesim2 to mimic turtlesim1.

\begin{lstlisting}[breaklines=true languages=xml]
</launch>
\end{lstlisting}

This closes the xml tag for the launch file.

\subsection{roslaunching}

Now let's roslaunch the launch file:

\begin{lstlisting}[breaklines=true languages=bash]
$ roslaunch beginner_tutorials turtlemimic.launch
\end{lstlisting}

Two turtlesims will start and in a new terminal send the rostopic command:

\begin{lstlisting}[breaklines=true languges=bash]
$ rostopic pub /turtlesim1/turtle1/cmd_vel geometry_msgs/Twist -r 1 -- '[2.0, 0.0, 0.0]' '[0.0, 0.0, -1.8]'
\end{lstlisting}
You will see the two turtlesims start moving even though the publish command is only being sent to turtlesim1.

\section{conclusion}

